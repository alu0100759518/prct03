\documentclass[a4paper,12pt]{article}
\usepackage[utf8]{inputenc}
\usepackage[spanish]{babel}
\begin{document}
\title{Título del artículo}
\author{Nombre y apellido \\
              Técnicas Experimentales~\footnote{Universidad de La Laguna}
              }
\date{\today}
\maketitle
\begin{abstract}
    En \LaTeX{} ~\cite{Lam:86} es sencillo escribir expresiones
    matemáticas como $a=\sum_{i=1} ^ {10} {z_i} ^{3}$
    y deben ser escritas entre dos símnbolos \$.
    Los superíndices se obtienen con el símbolo \^{}, y
    los subíndices con el símbolos \_.
    por ejemplo: $x^2 \times y ^{\alpha + \beta}$
    También se pueden escribir fórmulas centradas:
     \[h^2=a^2 + b^2 \]
\end{abstract}

\section{Primera sección}


Si simplemente se desae escribir un texto normal en LaTeX
sin complicadas f\'ormulas matem\'aticas o efectos especiales
como cambios de fuente, entonces simplemente tiene que escribir
en espa\~nol normalmente.\par 
Si desea cambiar de párrafo ha de dejar una linea en blanco o bien
utilizar el comando $\\par$ \par 
No es necesario peocuparte de la sangría de los párrafos
todos los párrafos se sangraran automaticamente con la excepción 
del primer párrafo de una sección.  
Se ha de dintinguir entre la comilla simple `izquierda' 
y la comilla simpe 'derecha' cuando se escribe en el ordenador
En el caso de que se quieran utilizar comillas dobles se han de escribir dos caracteres de `comillas simples' seguidos, esto  es, 
'' comillas dobles'' .
Tambien se ha de tener cuidado con los guiones: se utiliza un único
guión para la separacion de silabas, mientras que se utilizan
tres guiones seguidos para producir un guión de los que se usan 
como signo de puntuación --- como en esta oración 

\bigskip
\begin{tabular}{|l|c|c|}
\hline
   Nombre & Edad & Nota \\ \hline
   Pepe   &   24 &   10 \\ \hline
   Juan   &   19 &    8 \\ \hline
   Luis   &   21 &    9 \\ \hline
\end{tabular}


\begin{thebibliography}{00}
   \bibitem{Lam:86}
     Lamport, Leslie.
     TLA in pictures.
     \emph{IEEE Transactions on Sofware Engineering},
      21(9), 768-775
      (1995)
\end{thebibliography}
\end{document}
